% !TeX root = ../main.tex

\chapter{\LaTeX\ 的標準文稿類別}
\label{ch:class}\index{文稿類別}

這章主要是在說明 \LaTeX{} 文稿的類別(document class\index{document class})\footnote{這在舊的版本稱為 style,這兩個詞意義上差不多,這些都是 \TeX{} 延伸出來的巨集定義,專門用來定義文章的大結構,以便簡化使用上及文稿的內容。},這是 \LaTeX{} 規範文稿整體結構的方法。使用 class 的用意,就是把版面結構處理和實際文稿分開,這樣的最大好處就是維持整篇文章排版結構上的一致性,也使文稿內容更清爽簡潔,使用者只要專心於文稿內容的寫作即可,如果 class 定義的好,也可以達到一文多變化又不變更文稿內容的目的,只要把引用的 class 換成別的就可以了,其他的可以不必更動。

目前,\LaTeX{} 有五種標準類別用於一般文件,可用於一般的書信、雜誌、期刊、報告及論文。但有些期刊、論文會要求一定的結構,這時得依需求另行訂定。因此,也有其他的類別存在,標準類別並不是唯一的。甚至,也可以自行撰寫自己的文稿類別。當然,我們一般使用是不需要這麼講究,這裡只介紹 \LaTeX{} 的標準類別。而且,如果有和他人交換文稿的需求時,我們應該盡可能的使用流通性較廣泛的類別。

另外,也有一些是關於 \LaTeX{} 說明文件及 macro\index{macro} 寫作時要用到的類別,這些已超出本篇文章的範圍。

\section{\LaTeX{} 類別的宣告}
\index{類別的宣告}

\LaTeX{} 的類別,要在文稿的一開頭時就宣告(當然,其上有註解是沒有關係的),他的一般格式如下:

\begin{quote}
   \begin{verbatim}
\documentclass[選擇性參數]{類別}
\end{verbatim}
\end{quote}
\index{documentclass@\verb=\documentclass=}

選擇性參數是可以省略的,但類別名稱則不能省,一定要指定一個類別。而且只能只有一個類別。

\section{類別的選擇性參數}

選擇性參數可以選擇多個,各個選項是以逗點分開的。

\begin{enumerate}
   \item {\ttfamily 10pt, 11pt, 12pt}\\
         指定內文一般正常字的大小,預設是 {\ttfamily 10pt}。其他點數沒有外來 package 的幫忙不能指定。
   \item {\ttfamily a4paper, letterpaper, b5paper, executivepaper, legalpaper}\\
         指定紙張大小,預設是 {\ttfamily letterpaper}。
   \item {\ttfamily fleqn} \\
         使數學式靠左對齊,預設是居中對齊。\index{fleqn@\texttt{fleqn}}
   \item {\ttfamily leqno} \\
         使數學式編號靠左,預設是靠右。
   \item {\ttfamily titlepage, notitlepage}\index{titlepage@\texttt{titlepage}}\index{notitlepage@\texttt{notitlepage}} \\
         決定 title page 是否獨佔一頁。預設 \texttt{article}\index{article@\texttt{article}} 不獨佔一頁,但 \texttt{report/book}\index{book@\texttt{book}}\index{report@\texttt{report}} 則會獨佔一頁,在這裡是可以指定來變更預設行為,例如 \texttt{article} 文稿,指定 \texttt{titlepage} 的話,那 title page 就會獨佔一頁。
   \item {\ttfamily onecolumn, twocolumn} \\
         文章單欄或兩欄式,預設是單欄,也就是不分欄。\index{onecolumn@\texttt{onecolumn}}\index{twocolumn@\texttt{twocolumn}}
   \item {\ttfamily twoside, oneside}\index{twoside@\texttt{twoside}}\index{oneside@\texttt{oneside}} \\
         是否區分奇偶數頁。預設 \texttt{article/report} 不區分,\texttt{book} 則會區分。一般的書籍,在裝訂的部份,他的中央線稱為書脊,偶數頁會在打開書籍時的左方,而且其中內容會偏向書脊的中央(此時是向右),反之,奇數頁會在右,內容一樣會偏左向書脊,在 \texttt{oneside} 的情形則不做這樣的區分,不管奇偶頁都會在紙張的中央部位。
   \item {\ttfamily landscape} \\
         橫向列印或縱向列印,預設縱向(portait)。\index{landscape@\texttt{landscape}}\index{portait@\texttt{portait}}
   \item {\ttfamily draft}\index{draft@\texttt{draft}} \\
         草稿式編譯,這時圖檔將不會被引入,可加快編譯的速度。不過,如果編譯是使用向量字型的話,編譯速度應該是還算很快。但使用 {\ttfamily draft} 的一個好處是,過長的地方會標示出來。
   \item {\ttfamily openright, openany}\index{openright@\texttt{openright}}\index{openany@\texttt{openany}} \\
         這是在控制,章的開始是否是奇數頁(right-hand page)\index{奇數頁(right-hand page)}。在 \texttt{book} 類別,預設章會從奇數頁開始,report 類別則不會。\texttt{article} 類別沒有章,所以,對此一設定會忽略。
\end{enumerate}

\section{類別的種類}

這裡只列出一般文章使用的類別,其他特殊巨集或 \LaTeX{} 正式文件所使用的類別就不列出了,一般使用,這些類別就足夠了。

\begin{quote}
   \begin{tabular}{>{\ttfamily }lll}
      類別    & 一般用途       & 特性                                           \\
      \hline
      article & 一般短文       & 無章,連續頁方式的安排,無奇偶數頁的區分       \\
      report  & 較長論文       & 章會起新頁,預設無奇偶數頁的區分               \\
      book    & 書籍類         & 章會於奇數頁起新頁,預設有偶數頁的區分         \\
      letter  & 信件           & 英文信件格式                                   \\
      slides  & 幻燈片         & 幾乎另用外來套件取代                           \\
      minimal & 測試及寫新類別 & 這是最簡單的類別,只規定了內文的寬、高,正常字
   \end{tabular}
\end{quote}
\index{article@\texttt{article}}\index{report@\texttt{report}}%
\index{book@\texttt{book}}\index{letter@\texttt{letter}}%
\index{slides@\texttt{slides}}\index{minimal@\texttt{minimal}}

當然,這些用途並不是固定不變的,得看使用者的安排,不想多花時間、精神的話,那就依 \LaTeX{} 預設的格式去使用,至少就不會太離譜。其中 \texttt{minimal} class 是用來測試用的,或者寫新的 class 用的,他完全沒有版面的安排,例如,沒有章節的結構,各種間距也沒有定義,預設的字型是正常字,沒有其他的變化,幾乎所有的變化要自行去定義。
