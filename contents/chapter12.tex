% !TeX root = ../main.tex

\chapter{後記}

大概簡單介紹了 \LaTeX{} 的使用,這樣夠用了嗎?很可能是不夠的,尤其是想變動風格的時候,但一般使用,不講求花俏,應該可以用了,剩下的只是熟練的問題。當然,許多細節可能並沒有說明清楚,但方向知道了,其他的自行查閱就行了。真碰到問題時,可以到 bbs/news 上詢問。

有許多沒有提及的東西,在這一章交代一下,有些是可以自行查閱的,有些則是尚未成熟,可能還不到真正實用的階段(也可能是我自己也不會用啦!:)。

\begin{enumerate}

  \item 微調 \\
  由於微調牽涉到對 \LaTeX{} macro 的一定程度的認識,因此並沒有說明得很清楚,理想的話應該先把 \TeX{}/\LaTeX{} 巨集的寫法先做簡單的介紹,這個部份可能另外專文介紹較妥,畢竟這篇文件是定位在入門級教材。

  \item 中文的處理 \\
  中文的處理還有很多模糊地帶,例如,索引、參考文獻及中文直排。目前的其他中文 \TeX{}/\LaTeX{} 也沒有介紹,例如 Chi\TeX{}、\texttt{cw}\TeX{} 及 PU\TeX{} 等等。

  \TeX{}/\LaTeX{} 系統的字型機制算是較複雜,安裝字型更是一般使用者的夢魘,英文寫作比較容易解決,通常系統上都會安裝好,中文的話就比較麻煩,除了詳細去介紹外,我們使用中文字型應該有個大家認同的規格才行,否則我這份文件拿到其他的中文 \TeX{}/\LaTeX{} 系統上的時候,就必須修改一下,至少換個字型名稱才能順利編譯。

  \item 實例嫌不夠完整 \\
  尤其是表格、圖形處理及數理排版的部份,並沒有交待得很完整。這裡頭當然牽涉到許多的背景知識的問題,不單純在排版本身,這在其他的排版系統一樣會碰到同樣的情況。當然,有很大的部份是我個人經驗不足的關係啦!:-)

  \item 各種檔案格式的介紹 \\
  \TeX{}/\LaTeX{} 系統中的檔案格式多如牛毛,包括一些中間產生的檔案都有他特定的目的,但我們並沒有多做介紹。原因是,這些檔案都牽涉到他的運作機制,這麼一來連運作機制也要說明才行,這樣會使篇幅大增,而且也會擾亂了初學者的學習步調,因此,只能留待往後有機會再來介紹。

  \item 現有巨集庫的整理 \\
  \LaTeX{} 的巨集實在是太多了,現有資料大多是英文的,是有必要整理出一份有系統的中文速查表,以免重複去製造輪子。

  \item 重音符號、歐洲字元 \\
  這些都沒有真正接觸到。這些內容,個人不敢造次,因為並不很熟悉,因此,得要有懂歐洲語系的朋友來個完整的介紹才行。

  \item Unicode 編碼文件的處理 \\
  這方面也沒介紹,但由於這是一篇入門級的 \LaTeX{} 教學文件,這方面的內容應該是由另外的專文來介紹可能會比較恰當。

  \item 投影片的製作 \\
  投影片、幻燈片的議題目前很流行,\LaTeX{} 也是可以製作精美的投影片,這方面的文件可以參考:

  \begin{quote}
    \url{http://www.miwie.org/presentations/presentations.html}
  \end{quote}

  當然,以上的資料是使用在英文語系,中文的話,我們得另做介紹。

  \item 和 XML/SGML/HTML 的配合 \\
  這個完全沒有提到,一方面 XML/SGML 的內容沒有想像中簡單,他們的應用範圍實在是太廣了,這方面的內容得慢慢來補充。

  \item 和資料庫系統的配合 \\
  這方面在目前 seaching 掛帥的 Internet 是相當重要的課題,這當然是不適合放在這份文件,得另外在進階的文件做更進一步的介紹。\TeX{} 系統的生命力、可塑性相當強,因此和其他的文件系統的結合能力也就比較容易達成。

  \item GUI 圖形界面 \\
  這個部份也沒有深入介紹,最重要是卡在中文的問題上,這方面需要有興趣的朋友共同來研究,雖然命令列環境的生產效率很高,但為了顧及一般使用者的習慣,方便入門的 GUI 圖形界面的確有其需要。

\end{enumerate}

這份文件,感謝行政院研考會委辦,朝陽大學洪朝貴授與輔仁大學毛慶禎教授共同主持的「政府機關資料文件交換之電子檔案格式應用研究」計畫做部份的補助,很高興他們都能認同自由文件、自由軟體,在目前社會體制下的存在價值。

這份文件的 PDF 格式所嵌入的中文字型,採用的是王漢宗博士所捐贈的三十幾套的 TTF 向量字型,王博士以前就曾捐贈過十三套的 TTF 向量字型,再加上這次的三十幾套,我們自由軟體社群的中文字型就更充實了。更難得的是,這些字型都是採用 GNU GPL 的授權,和這份 GNU FDL 自由文件配合起來,相得益彰,感謝王漢宗博士的慷慨無私及對自由軟體、自由文件的認同。

這份文件採用的是 PDF 及 HTML 格式,這當然得需要網路資源才能呈現在各位面前,感謝 kenduest 及交大數學和 ctshieh 及淡江數學提供這方面的資源,沒有他們的幫忙,文件就無法呈現在各位眼前了!

也希望大家能對這份文件多多的指正及建議。寫作期間,已經接到許多朋友的來信指正,很感謝他們。這份文件的 HTML/PDF 格式及原始文稿,可以在以下網站取得:

\begin{quote}
  \url{http://edt1023.sayya.org/tex/latex123/index.html} \\
  \url{http://edt1023.sayya.org/tex/latex123/latex123.pdf} \\
  \url{http://edt1023.sayya.org/tex/latex123/latex123-v1.0-src.tar.gz} \\
  \url{http://MathNet.math.tku.edu.tw/~edt1023/tex/latex123/index.html} \\
  \url{http://MathNet.math.tku.edu.tw/~edt1023/tex/latex123/latex123.pdf} \\
  \url{http://MathNet.math.tku.edu.tw/~edt1023/tex/latex123/latex123-v1.0-src.tar.gz}
\end{quote}
