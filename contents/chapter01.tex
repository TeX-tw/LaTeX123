% !TeX root = ../main.tex

\chapter{先來說一些故事}

\TeX{} 是 Donald~E.~Knuth\footnote{\url{http://www-cs-faculty.stanford.edu/~knuth/}}\index{Donald~E.~Knuth} 教授的精心傑作,它是個功能非常強大的幕後排版系統,含有彈性很大,而且很低階的排版語言。當初,是因為 Knuth 教授在寫他的大著 TAOCP(\textit{The Art of Computer Programming})\index{TAOCP} 時,發覺書商把他書中的數學式子排得太難看了,於是決定自行開發一個非常適合排數學式子的排版語言,這就是 \TeX{} 系統的來由。

不僅僅是談到 \TeX{} 一定會提到 Knuth教授,只要提到排版,沒有人可以忽略他的 \TeX{} 所帶來的變革、影響,甚至 \TeX{} 已經 20 幾歲了,仍然深深影響著排版界及排版軟體,可見這個排版軟體真的是非同小可。

\section{Knuth 教授的生平簡介}

\begin{quote}
\begin{tabular}{lp{.78\textwidth}}
1938.01.10  & 誕生。Milwaukee, Wisconsin; U.S. citizen。\\
1956        & 進入凱思工學院(Case Institute of Technology)學習物理。\\
1960        & 畢業後進入加州理工學院(California Institute of Technology)研究所,此時轉向數學領域的研究方向。\\
1961.06.24  & 和 Nancy Jill Carter 結婚。他的中文名字是高德納\index{高德納},他老婆名叫高精蘭\index{高精蘭},老婆小他一歲。兩個小孩,一男一女。中文名字是 1977 去中國大陸時取的。\\
1963        & 拿到 Ph.D.,並留校任教。\\
1968        & 開始任教於 Stanford 大學,資訊科學系(Computer Science)。同年開始撰寫名聞遐邇的 TAOCP({\em The Art of Computer Programming})。有人曾說,看了這部書,往後對寫程式的話題都會變得謙虛。:)\\
1977        & 不滿意書商所印出的 TAOCP,因此,自行開發 \TeX{} 排版系統,這可就影響了往後的排版、出版界,至今不墜。但也因此拖延了 TAOCP 第四冊的完成時間。\\
1986        & 榮獲 ACM 軟體系統獎。
\end{tabular}
\end{quote}

他可說是著作等身,書籍、論文都有,他的任何著作有個奇怪的『副作用』,那就是任何人發現書上的錯誤,都可以向他舉發,並領取 \$2.56(美金)!想試試看「手氣」嗎?台灣就有人領過。:-) 為什麼是 \$2.56?Knuth 教授的答案是:

\begin{quote}
\itshape
``256 pennies is one hexadecimal dollar.''
\end{quote}

發現 \TeX{} 系統的臭蟲也是一樣,這個獎金每年倍數增加,直至 \$655.36($2^{16}$ pennies) 為止。

他也很推崇自由軟體基金會 (Free Software Foundation)%
\footnote{\url{http://www.fsf.org}}\index{自由軟體基金會}\index{Free Software Foundation}
及 GNU/Linux\index{GNU/Linux},把一些希望都寄託於 GNU/Linux,尤其是 Unicode 環境,他希望 GNU/Linux 很快就能在網頁上顯示他的中文名字,而不必使用圖檔。其實是可以做到了,只是 Unicode 環境還不算普及罷了。他曾在 1996 年接受 Dr.~Dobb's~Journal 訪問時英雄惜英雄的公開表示,創導自由軟體 (Free Software)%
\footnote{\url{http://www.gnu.org/philosophy/free-sw.html}}\index{自由軟體}\index{Free Software}
的 Richard~M.~Stallman\index{Richard~M.~Stallman} 是他心目中的英雄之一,他認為自由軟體基金會這些人所做的貢獻很不錯,雖然和他的方式不一定一樣,但許多貢獻是互有認同的。

在發展 \TeX{} 時就同時思考 WEB(這個詞比目前使用的 WWW Web 還早使用)\index{WEB},那是一種 literate programming\index{literate programming} 的程式方法。他認為目前已成熟的可以提出含有文件的程式方法,使寫程式就像寫文學作品(小說?)一般的藝術表現。後來也把他由 C 改寫(和 Silvio~Levy\index{Silvio~Levy} 合作),名為 CWEB\index{CWEB}\index{CWEB}%
\footnote{\url{http://www-cs-faculty.stanford.edu/~knuth/cweb.html}}。\TeX{} 就是由 WEB 寫成的,WEB 可視為 Pascal 語言的一個子集。

一個 literate 程式師可被視為文學作家、評論寫作者、隨筆作者\chdots{},程式的表現不僅僅是搬弄符號,而是展現自己的風格,當然也是指達成目的的風格、甚至程式中變數運用的風格。

這樣一來就可以展現讓人類較能理解的程式碼。使用形式及非形式的融合,而且兩者間相輔相成,目的達成了,也讓閱讀的人就好像讀文學作品般的去抓住作者的心,使程式創作提升至更高的(文學)藝術境界,而不再是死板板的 code 了。

Knuth 大師已於 1992 年自大學退休,但仍在 Stanford/Oxford 等大學有授課。目前正處於隱居的生活,他這麼早退休的原因,就是因為 TAOCP 這部書,他估計大約要花 20 年來完成,因此目前的重點工作是完成他的 TAOCP(分成好幾冊,目前真正出版的只有三大冊)。他認為
email\footnote{\url{http://www-cs-faculty.stanford.edu/~knuth/email.html}}
會影響他的思路,因此,寧願留住址,要和他聯絡就只好寫信,傳真。給他的秘書的 email,是最後有時間才會去看,他曾公開的表明,這部書是他這一生中最重要的工作。

雖然 Knuth 教授寫了許多嚴肅艱深的書籍、論文,但是他也是有風趣的一面,在 1996 年,\textit{Mathematisch Centrum }(MC, 為慶祝五十周年慶改稱為 \textit{Centrum voor Wiskunde en Informatica}, CWI) 曾邀請他演講,並知會荷蘭的 \TeX{} User Group(NTG)%
\footnote{\url{http://www.ntg.nl},荷蘭的 \TeX{} User Group 算是相當活躍的。},NTG 見機會難得,就邀請 Knuth 教授另開個 \TeX{} 及 \MF\ 討論會,並接受大家的提問,他說:『不對,我也是可以問你們問題的!』。而且,他還說:『這種問答的內容,很可能在不同場合重複過,所以,如果我對同一個問題,曾有過兩種答案的話,你們必需取其平均值。』他的妙語如珠,在許多類似的場合常常引起哄堂大笑,但實際的內容卻絕非泛泛之言。:-)

\subsection{TAOCP(\textit{The Art of Computer Programming})\index{TAOCP}\index{The Art of Computer Programming@\textit{The Art of Computer Programming}}}

這可是演算法的大著,請別折騰我,我只是心嚮往之,這部書我沒有一本是看得懂的。:-)

\begin{quote}
  \begin{enumerate}
    \item 第一冊,{\em Fundamental Algorithms},1968 第一版,ISBN 0-201-89683-4。
    \item 第二冊,{\em Seminumerical Algorithms},1969 第一版,ISBN 0-201-89684-2。
    \item 第三冊,{\em Sorting and Searching},1973 第一版,ISBN 0-201-89685-0。
    \item 第四冊,{\em Combinatorial Algorithms},尚未完成,可能會先出分冊。
      \begin{enumerate}
        \item 分冊 4A, {\em Enumeration and Backtracking}
        \item 分冊 4B, {\em Graph and Network Algorithms}
        \item 分冊 4C, {\em Optimization and Recursion}
      \end{enumerate}
    \item 第五冊,{\em Syntactic Algorithms},計劃 2010 年完成。
    \item 第六冊,the theory of context-free languages,書名可能會變更,尚未開始。
    \item 第七冊,Compiler techniques,書名可能會變更,尚未開始。
  \end{enumerate}
\end{quote}

詳細的目錄大綱及修訂版的資訊,請參考網頁上的資料:

\begin{quote}
  \url{http://www-cs-staff.stanford.edu/~knuth/taocp.html}
\end{quote}

\section{那麼,\LaTeX{} 又是什麼呢?}

\TeX{} 是個很低階的排版語言,如果排版時都要從這種低階語言來控制版面的話,那將會非常的煩複,所以,一些經常要用到的功能,都會先去定義好(稱為巨集,macro\index{macro}\index{巨集}),這樣排版時才會方便、快速,直接引用已定義好的巨集裡頭的指令就可以了。

原始的 \TeX{} 已經有了一組 macro,是 Knuth 教授所寫的,那就是著名的 Plain \TeX{}\index{Plain \TeX{}},但仍然不夠方便、直觀,於是
Leslie~Lamport\index{Leslie~Lamport}\footnote{Leslie~Lamport 於 1985 在
\htmladdnormallink{CSL}{http://www.csl.sri.com/}
任職時寫了 \LaTeX{}。目前任職於
\htmladdnormallink{DEC Systems Research Center}{http://www.research.digital.com/SRC/home.html}
但已幾乎沒有參與 \LaTeX{} 的後續發展了。}
又寫了另一組的 macro,稱為 \LaTeX{},主要是把版面配置和文章內容,適度的分開處理,只要使用者選定了一種類別,整本書或整篇文章的結構就是按照這個類別來安排版面,這樣寫文件的人只要專注於文章內容就可以了,版面配置就完全交給 \TeX{}/\LaTeX{} 去處理。

既然 \LaTeX{} 只不過是 \TeX{} 的一大組巨集,那,當然原來的 \TeX{} 的指令,大部份也是可以用在 \LaTeX{} 文稿當中的。而且,\LaTeX{} 並不是目前唯一的 \TeX{} macro,其他如 eplain \TeX{}, Con\TeX{}t, \TeX{}info 等都是 \TeX{} macro,也都有他們自成一套的語法。目前的 \LaTeX{} 由
\LaTeX{} 3 Project\footnote{\url{http://www.latex-project.org/latex3.html}}
所維護及發展。

如果談到這裡,你還是霧煞煞的話,類比成 HTML markup 標記語言就能大概知道一些概念了,當然,這和 HTML 標記語言是可相提而無法並論的。如果連 HTML 也不熟悉,那也沒關係啦!這章本來就是在說故事嘛!:-) 只要繼續看後面的內容就行了。

\section{一般人對 \LaTeX{} 的迷思}

這裡引用 Peter Flynn\index{Peter Flynn} 在他的
\textit{A beginner's introduction to typesetting with \LaTeX{}}\footnote{\url{ftp://ftp.dante.de/tex-archive/info/beginlatex/beginlatex.pdf}}
一文中所提出來的六大迷思,並添加個人的一些看法:

\begin{itemize}

  \item \LaTeX{} 只能使用一種字型? \newline
  當然不是,雖然 \TeX{} 系統預設是使用當初 Knuth 教授所設計的 \MF{},但目前 OpenType, TrueType, Adobe Type1 字型都可以用在 \LaTeX{} 當中,只不過,安裝字型的部份不是那麼的直觀就是了,但比起其他的排版系統,\TeX{}/\LaTeX{} 所能利用的字型種類,可以說是最多的。

  \item \LaTeX{} 只能用於 Unix-like 的作業系統? \newline
  Knuth 教授慷慨的把 \TeX{} 的原始程式碼開放出來\footnote{雖然有一些版權的規定,但是 Knuth 教授的本意,這些原始碼是屬於 Public Domain,只是,如果經過修改,不能再以 \TeX{} 為名,要另外改一個名稱,以免和原來的 \TeX{} 搞混。\TeX{} 這個商標的所有人是美國數學協會(American Mathematical Society, AMS\index{AMS}\index{American Mathematical Society})。},所以,只要是有人在使用的作業系統都可以移植過去,不必擔心版權的問題。像 MS DOS, OS/2, MS Windows, Mac OS, Unix-like 系統都有 \TeX{} 的移植版本,甚至是 PDA(e.g.~the Sharp \textsl{Zaurus})都有 \TeX{}/\LaTeX{} 的縱跡。可以說是走到哪兒,就可以用到哪兒,而且,文稿都是互通的,列印結果也相同。

  \item \LaTeX{} 已經過時了? \newline
  剛好相反,\LaTeX{} Project\footnote{\url{http://www.latex-project.org/}}及其他相關 packages 正穩定的研發當中,尤其和目前新一代的 SGML/XML/HTML 及資料庫系統,都積極的想辦法銜接起來。對於數學式子的排版,至今無人能出其右。有興趣的話,可以參考 \url{news://comp.text.tex} 的流量,及其中 \textsc{CTAN}\footnote{\url{http://tug.ctan.org/}}\index{CTAN} 對於巨集更新、上傳的消息發布。

  \item \LaTeX{} 沒有所見即所得(WYSIWYG\index{WYSIWYG})? \newline
  某種層面上而言,是的。因為他本質上是幕後排版系統\index{幕後排版系統}。但是,所產生的 dvi/ps/pdf 檔,可以很精準的顯示你所想要表達的內容,不曉得這算不算是「所見即所得」?另外,一些相關 GUI 配合的軟體,也會打破幕後排版的定義,例如 LyX\footnote{\url{http://www.lyx.org/}}\index{LyX}, \TeX{}macs\footnote{\url{http://www.texmacs.org/}}\index{TeXmacs@\TeX{}macs} 等等。

  \item \LaTeX{} 很難學? \newline
  這個嘛!我只能說,有什麼東西是很好學的?如果只是一般使用,而不是當個排版專家,甚至 \TeX{}/\LaTeX{} programmer,那麼,幾十分鐘的說明,就可以「上工」了!剩下的只是一些細部調整的問題(不去調整,也絕對不會離譜)。相對於一般圖形化 Office 類軟體,要真正把他的內容熟悉,恐怕也是不簡單的。另外的問題,大概是幕前、幕後系統操作習慣的問題,甚至是一種第一印象了,就好像說到電腦,有不少人的腦子裡就會浮現 MS Windows 的圖象一樣。:-)

  \item \LaTeX{} 是專為科學家或數學家而寫的? \newline
  的確,當初 Knuth 教授是為了表達精確、品質優美的數學式子而開發 \TeX{} 的,但由於 \TeX{} 的彈性,使得在其他的領域的使用者也爭相使用,已經不是局限在學術界在使用了。尤其 XML\index{XML} 的興起,需要一個適合的格式化工具(formatter\index{formatter})的配合,\TeX/\LaTeX{} 就剛好稱職的做他排版專業的工作。

\end{itemize}

\section{本文的重點方向}

\TeX{}/\LaTeX{} 的相關議題:版面的配置、排版規範、字型技術、\TeX{} the program 的改進、繪圖技術、\TeX{} macro 的撰寫、pre/post 處理程式的撰寫、出版流程 \chdots{} 等等,浩瀚無涯,可以說窮一輩子也可能研究不完,所以,各位在「陷入」這個領域之前,建議最好有個適當的範圍,以免「愈陷愈深」終至無法自拔。

所以,本文的重點是放在「標準」 \LaTeX{} 本身,其他相關的 package/macro 除非必要,盡量不提及。但是 \LaTeX{} 本身也不是十全十美的,所以,有需要的地方也需要一並提及外來的 macro,無論如何,重點是放在標準 \LaTeX{} 本身,請閱讀本文的朋友注意一下這個方向。\LaTeX{} 本身就能解決的,就不假外求了,雖然會失去了一些彈性,但為免造成 package/macro 滿天飛,打亂學習陣腳,剛開始也只好如此了!

而且,可能的情形下,會往一般用途的方向去介紹,而不僅僅專注在數理排版。數學式子雖是 \TeX{}/\LaTeX{} 的拿手把戲,但不代表一般用途就不適合,相反的,現在有許多商業公司正把 \TeX{}/\LaTeX{} 用於一般商業出版上。
