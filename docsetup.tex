% --------------------------------------------------
% 加載套件(Include Packages)
% --------------------------------------------------

\usepackage[CJKbookmarks, plainpages=false, hyperindex]{hyperref}
\usepackage{mflogo}
\usepackage{amsmath, amsthm, amssymb}
\usepackage{array}
\usepackage{alltt}
\usepackage{fancyvrb}
\usepackage{makeidx}
% \usepackage{imakeidx}
% \usepackage[pagecolor={none}]{pagecolor}

% --------------------------------------------------
% 套件設定(Packages Settings)
% --------------------------------------------------

% \appendgraphicspath{{./figures/chapter01/}}
% \appendgraphicspath{{./figures/chapter02/}}
\appendgraphicspath{{./figures/chapter03/}}
\appendgraphicspath{{./figures/chapter04/}}
\appendgraphicspath{{./figures/chapter05/}}
\appendgraphicspath{{./figures/chapter10/}}

% --------------------------------------------------
% 自訂命令(Packages Settings)
% --------------------------------------------------

\hypersetup{
    colorlinks  = true,
    linkcolor   = blue,
    pdfauthor   = {李果正 Edward G.J. Lee},
    pdftitle    = {大家來學 LaTeX Version 1.0},
    pdfsubject  = {大家來學 LaTeX Version 1.0},
    pdfkeywords = {大家來學 LaTeX, CJK, LaTeX, latex123}
}

% 設定小節深度
\setcounter{secnumdepth}{3}

% 啟用頁眉格式
\pagestyle{fancy}

% 繪製線段長度
\newcommand{\drawwidth}[1]{\rule{0.3pt}{1.3ex}\rule[1mm]{#1}{0.3pt}\rule{0.3pt}{1.3ex}}

% 中文顯示日期
\renewcommand{\today}{\number\year~年~\number\month~月~\number\day~日}

% 中文刪節符號
\newcommand{\chdots}{$\cdots\cdots$}
\newtheorem{defi}{Definition}

% 設置中文標題
\renewcommand\contentsname{目~錄~}
\renewcommand\bibname{參~考~資~料}
\renewcommand\indexname{索~引}
\renewcommand\chaptermark[1]{\markboth{第\ \thechapter\ 章\  #1}{}}

% 創建索引頁面
\let\oldprintindex\printindex
\renewcommand{\printindex}{
  \addcontentsline{toc}{chapter}{索引}
  \oldprintindex
}
\makeindex